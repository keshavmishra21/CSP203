\documentclass{article}
\usepackage{amsmath}
\begin{document}
	\noindent\textbf{Exercise 3.2.} What is an example of a proof with mathematical content?\\
	
	\noindent\textbf{Solution:}\\
	
	\noindent\textit{Proof.} If you want to include math in a sentence, you use \$. For example (see latex file), $ \int x\, = \frac{1}{2}x^2 + C $. If you want to display math (centered on a new line), use \$\$. For example (see latex file),
	\[
	\sum_{i=1}^{100} i = 5050
	\]
	Next is an example of the \textbf{align} environment:
	\begin{align*}
		\sum_{i=1}^{k+1} i &= \left(\sum_{i=1}^{k} i \right) + (k + 1)\\
		                   &= \frac{k(k + 1)}{2} + k +1 \quad \hspace{2cm} \text{(by inductive hypothesis)} \\
		                   &= \frac{k(k + 1) + 2(k + 1)}{2}\\
		                   &= \frac{(k + 1)(k + 2)}{2}\\
		                   &= \frac{(k + 1)((k + 1) + 1)}{2}
	\end{align*}
\end{document}